\documentclass{article}
\usepackage{amsthm}
\usepackage{amsmath}
\usepackage{amssymb}
\usepackage{multicol}
\usepackage{graphicx}
\usepackage{epstopdf}

\theoremstyle{definition}
\newtheorem{mydef}{Definition}
\newtheorem{myex}{Example}
\newtheorem{mythe}{Theorem}
\newtheorem{mylemma}{Lemma}
\newtheorem{mycor}{Corollary}
\newtheorem*{myab}{Abstract}
\begin{document}

\title{Reachability Notes}
\maketitle

\section{What is reachability?}

To convert from $\mathcal{G}$ to $\mathcal{G}^u$, we use the following:
\\
1. $V_i \rightarrow V_j$ in $\mathcal{G}^u$ iff $\exists$ a directed sequence from $V_i$ to $V_j$ of length $u$ in $\mathcal{G}$
\\
2. $V_i \leftrightarrow V_j$ in $\mathcal{G}^u$ iff $\exists$ a pair of directed sequences:
\\
$ \pi_1$:$ V_c \rightarrow ... \rightarrow V_i$
\\
$ \pi_2$:$ V_c \rightarrow ... \rightarrow V_j$
\\
where length($\pi_1$)=length($\pi_2$) $\leq u$ and $V_c$ is some vertex in $\mathcal{G}$ 
\\
\\
The pair of directed sequences in step $2$ forms a \textit{fork}. Also $\mathcal{G}= \mathcal{G}^1$. No bidirected edges are allowed in $\mathcal{G}^1$. Also note that if $u \rightarrow v$ and $v \rightarrow u$, this is not the same as $u \leftrightarrow v$. Sergey considers these separate. The former is two single directed edges whereas the latter is one bidirected edge. It can be the case that there exists a bidirected edge between $u$ and $v$: $u \leftrightarrow v$ and only a single directed edge from $u$ to $v$: $u \rightarrow v$.
\\
\\
Converting from $\mathcal{G}^1$ to $\mathcal{G}^u$ is called the \textit{forward problem}. Finding matching $\mathcal{G}^1$'s given a $\mathcal{G}^u$ is called the \textit{backward problem}. 
\\
\\
Although we can extend $u$ indefinitely for $\mathcal{G}$, there exists a $u_{MAX}$ such that for any $u \geq u_{MAX}$, $\mathcal{G}^u = \mathcal{G}^{u_{MAX}}$. This often happens when we find a $u$ that makes $\mathcal{G}^u$ a superclique, a graph where every pair of distinct vertices is connected by a directed edge. Increasing $u$ further cannot add any additional edges since all possible edges are already in $\mathcal{G}^u$ that is a superclique. 
\\
\\
Let's make $f$ be a function that takes as input a graph $\mathcal{G}$ and outputs a set $\{\mathcal{G}^2,\mathcal{G}^3,...,\mathcal{G}^{u_{MAX}}\}$. Let us consider the codomain of $f$. Assuming that $\mathcal{G}$ has $n$ many edges, we will have ${n \choose 2}$ many possible bidirected edges and $n^2$ many possible directed edges. Thus, the cardinality of the codomain is $2^{n^2+{n \choose 2}}$. However, we are only interested in the range. We call a graph in $range(f)$ \textit{reachable}. Suppose we have an unreachable graph $\mathcal{H} \in codomain(f)$. The \textit{nearest reachable} graph from $\mathcal{H}$ is a reachable graph $\mathcal{H'} \in range(f)$ that requries the least number of edge additions/reversals/deletions from $\mathcal{H}$.

\newpage

\section{Questions}

Suppose the ground truth $\mathcal{G}$ is unknown to us. The questions we wish to investigate are:
\\
\\
1. What properties distinguish reachable and unreachable graphs? 
\\
2. Given some unreachable graph $\mathcal{H}$ how do we obtain the nearest reachable graph?




	


%\begin{figure}[!h]
%\includegraphics[scale=.28]{name_of_pic}
%\end{figure}


\end{document}
%$\mathcal{G}^u$